\documentclass{article}
\usepackage[margin=0.7in]{geometry}
\usepackage{amsmath,amssymb,amsfonts,amsthm}

\begin{document}
\title{J. R. "Bob" Dobbs Memorial Inverse Hessian-Vector Multiplication Method}
\author{Howon Lee}
\date{2019 January}
\maketitle

\begin{quote}
I'm not saying that machine learning is the portal to a demon universe, I'm just saying that some doors are best left unopened. - James Mickens
\end{quote}

\section{Abstract}
You care about multiplication of the inverse Hessian with a vector because Newton's method for optimization demands it. In many applications, inverse Hessian multiplication is not possible because of time and space constraints. A method linear to the size of the gradient function in both time and space is given which is very close to the Hessian-vector multiplication of \cite{pearlmutter1994}, with differential operator and all. However, I have failed to get it working on neural nets for a reason also given. I also have some strange speculative statements.

\section{Introduction}

The Hessian is the matrix of second derivatives of a function. In Newton's method for optimizations, it is the second derivative with respect to an optimization target. But Newton's method requires the multiplicative inverse of the Hessian (in one dimension, it requires the multiplicative inverse of the second derivative). This led to Newton's method being computationally infeasible for most optimizational regimes where the dimensionality is high, because if the dimension of the system is $n$, the naive inversion of the Newton's method takes $O(n^2)$ space and $O(n^3)$ time.

There is a shortcut which already exists which is usually more used in neural networks. In neural networks it was introduced by \cite{pearlmutter1994}. It consists in noting that the Hessian is a Jacobian itself and finding the Hessian in the Taylor approximation of the gradient about a point. This means that the Hessian-vector multiplication can be found like a sparse matrix-vector multiplication, linear to the order of the vector. However this gives a Hessian-vector multiplication only, not an inverse Hessian-vector multiplication, so one must use Krylov methods to get optimization done\cite{martens2010}. In the following, an inverse Hessian-vector multiplication is shown.

\section{J. R. "Bob" Dobbs Memorial Method}

Recall the Hessian is a Jacobian, and it is found in the Taylor expansion of the gradient of a function about a point. The crux of the method, almost trivial as it is, is to realize that the inverse Hessian is itself a Jacobian and it is found in the Taylor expansion of the functional inverse of the gradient of a function about a point.

\subsection{Inverse Hessian is a Jacobian}

The inverse function theorem \cite{lang1995} gives a formula for the matrix inverse of a Jacobian, with conditions (Jacobian determinant must be nonzero). That is, the Jacobian of the (functional) inverse of a function is the multiplicative inverse of the Jacobian of the original function. There is no guarantee for the whole domain.

The Hessian is the Jacobian of the gradient:

$$H(f(x)) = J(\nabla f(x))^T $$

So, the multiplicative inverse of the Hessian is the Jacobian of the functional inverse of the gradient:

$$H^{-1}(f(x)) = J(\nabla^{-1} f(x))^T $$

\subsection{Inverse Hessian is found in the Taylor Expansion of $\nabla_w^{-1}$}

The method of \cite{pearlmutter1994} for Hessian-vector multiplication begins from the Taylor expansion of the gradient about a point:

$$\nabla_w (w + \Delta w) = \nabla_w (w) + H\Delta w + O(||\Delta w||^2)$$

Where the $H$ appears because the Hessian is the Jacobian of the gradient, assuming Schwartz symmetry of second derivatives.

Take $rv = \Delta w$ and isolate the Hessian:

$$H(rv) = rHv = \nabla_w(w + rv) - \nabla_w(w) + O(r^2)$$

$$Hv = \frac{\nabla_w(w + rv) - \nabla_w(w)}{r} + O(r)$$

Take limit as $r \rightarrow 0$:

$$Hv = \frac{\partial}{\partial r} \nabla_w (w + rv) |_{r=0}$$

Pearlmutter calculates that by defining a differential operator $R$ (the G\^{a}teaux derivative\cite{gateaux1913}) such that:

$$R(f(w)) = \frac{\partial}{\partial r} f(w + rv)|_{r=0}$$

and mechanically going through the equations for calculating $\nabla_w$ step by step, so that the $R(\delta)$ get cached there, too.

But, given our definition of $H^{-1}$ above, we will actually find $H^{-1}$ about the expansion of the functional inverse of the gradient about a point. Not the integral of the Green's function (at least not directly), not the multiplicative inverse of the gradient (unless the multiplicative inverse corresponds with the functional inverse), the functional inverse.

$$\nabla_w^{-1} (\nabla_w + \Delta w) = \nabla_w^{-1} (\nabla_w) + H^{-1} \Delta w + O(||\Delta w||^2)$$

After finding this, the entire rest of the procedure goes analogously to \cite{pearlmutter1994}.

$$H^{-1}(rv) = rH^{-1}v = \nabla_w^{-1}(\nabla w + rv) - \nabla_w^{-1}(\nabla w) + O(r^2)$$

$$H^{-1}v = \frac{\nabla_w^{-1}(\nabla w + rv) - \nabla_w^{-1}(\nabla w)}{r} + O(r)$$

$$H^{-1}v = \frac{\partial}{\partial r} \nabla_w^{-1} (\nabla w + rv) |_{r=0}$$

And the differential operator, denoted $B$ because of the different semantics, is as such. Note it's still just a G\^{a}teaux derivative.

$$B(f(\nabla w)) = \frac{\partial}{\partial r} f(\nabla w + rv)|_{r=0}$$

\section{Examples}

\subsection{}

$$ f_1(x) = \sum_i x_i^3 $$

$$ \nabla f_1(x)_i = 3 x_i^2 $$

For reference, $Hv$.

$$ Hv = \frac{\partial}{\partial r} 3 (x_i + rv_i)^2 |_{r=0} $$

$$ Hv = 6x_iv_i \frac{\partial}{\partial r} 3 (x_i + rv_i)^2 |_{r=0} $$

Let $ \nabla f_1(x)_i = z_i$

$$ x_i = \sqrt{\frac{z_i}{3}} $$

$$ H^{-1}v = \frac{\partial}{\partial r} (\frac{z_i + rv_i}{3})^{\frac{1}{2}} |_{r=0} $$

$$ H^{-1}v = \frac{1}{2} (\frac{z_i + rv_i}{3})^{-\frac{1}{2}}(\frac{v_i}{3}) |_{r=0} $$

$$ H^{-1}v = \frac{1}{2} (\frac{z_i}{3})^{-\frac{1}{2}}(\frac{v_i}{3}) $$

Recall $z_i = 3 x_i^2$ if you want it in terms of $x$.

Of course, that one was trivial anyhow. Let's do one with some off-diagonal elements in the Hessian.

\subsection{}

$$ f_2(x) = \prod_i x_i$$

Take cardinality of the x vector to be 3 to avoid extremely long derivation.

$$ f_2(x) = x_1x_2x_3 $$

$$ \nabla f_2(x)_1 = x_2x_3 $$
$$ \nabla f_2(x)_2 = x_1x_3 $$
$$ \nabla f_2(x)_3 = x_1x_2 $$

For referene, $Hv$.
$$ (Hv)_1 = \frac{\partial}{\partial r} (x_2 + rv_2)(x_3 + rv_3) |_{r=0} $$
$$ (Hv)_1 = v_2x_3 + v_3x_2 + 2rv_2v_3 |_{r=0} $$
$$ (Hv)_1 = v_2x_3 + v_3x_2 $$
$$ (Hv)_2 = v_1x_3 + v_3x_1 $$
$$ (Hv)_3 = v_1x_2 + v_2x_1 $$

Again let $ \nabla f_2(x)_i = z_i$

$$ x_1 = \sqrt{\frac{z_2z_3}{z_1}} $$
$$ x_2 = \sqrt{\frac{z_1z_3}{z_2}} $$
$$ x_3 = \sqrt{\frac{z_1z_2}{z_3}} $$

$$ (H^{-1}v)_1 = \frac{\partial}{\partial r} (\frac{(z_2 + rv_2)(z_3 +rv_3)}{(z_1 + rv_1)})^{\frac{1}{2}} |_{r=0} $$
$$ (H^{-1}v)_1 = (\frac{(z_2 + rv_2)(z_3 +rv_3)}{(z_1 + rv_1)})^{-\frac{1}{2}} \frac{(v_2(z_3 + rv_3) + v_3(z_2 + rv_2))(z_1 + rv_1) - v_1(z_2 + rv_2)(z_3 + rv_3)}{(z_1 + rv_1)^2} |_{r=0} $$
$$ (H^{-1}v)_1 = (\frac{(z_2)(z_3)}{(z_1)})^{-\frac{1}{2}} \frac{(v_2z_3 + v_3z_2)(z_1) - v_1z_2z_3}{z_1^2}$$
$$ (H^{-1}v)_2 = (\frac{(z_1)(z_3)}{(z_2)})^{-\frac{1}{2}} \frac{(v_1z_3 + v_3z_1)(z_2) - v_2z_1z_3}{z_2^2}$$
$$ (H^{-1}v)_3 = (\frac{(z_1)(z_2)}{(z_3)})^{-\frac{1}{2}} \frac{(v_1z_2 + v_2z_1)(z_3) - v_3z_1z_2}{z_3^2}$$

So you can see here that the difficult part is finding a functional inverse for the gradient, since after that the operator is completely mechanical, if rather tedious. Again, you can also have it in terms of $x_i$.

\section{Neural Net}

Of particular interest in application of Newton's method to high dimensional systems is in neural networks. I have made several attempts to attack them with this method but it is surprisingly difficult to get the functional inverse of a neural network gradient with respect to the weights which is compatible with the differential operator.

With minibatches (but only with very large minibatches), the functional inverse of the gradient is actually pretty trivial. However, with actual application of the differential operator, I found that unlike the DAG structure of the dependencies of the original Pearlmutter application of the operator, initial results require further results in the inverse regime. Of course one could just approximate those numerically, so at least I present what I have right now for a 1-hidden-layer fully-connected multilayer perceptron. (You can also use the approximation of the G\^{a}teaux derivative which doesn't take the limit of $r$ pretty easily, but that one has numerical problems.)

The G\^{a}teaux derivative is a formalization of the functional derivative, but I haven't gotten anywhere by using the functional derivative, although it seems tempting. Of especial note is the fact that

$$\frac{\delta f^{-1}(x)}{\delta f(y)} = - \frac{\delta(f^{-1}(x) - y)}{f'(f^{-1}(x)} $$

But then you have to find the inverse of the neural net in the first place. The same holds for finding the Legendre transformation.

Often the Jacobian determinant is zero, but you can avoid this by construction (pick another random initialization). I assume you are completely familiar with neural nets already\cite{deeplearning}.

Specific example is with sum of squared error and no softmax.

Forward steps of the plain neural net:

$$ net_1 = xW_1 $$
$$ h_1 = act(net_1) $$
$$ net_2 = h_1W_2 $$
$$ h_2 = act(net_2) $$
$$ J = err(h_2, y) $$

Backward steps:

$$ \frac{\partial J}{\partial h_2} = h_2 - y $$
$$ \frac{\partial J}{\partial net_2} = act'(net_2)\frac{\partial J}{\partial h_2}$$
$$ \frac{\partial J}{\partial h_1} = \frac{\partial J}{\partial net_2} W_2^T$$
$$ \frac{\partial J}{\partial net_1} = act'(net_1)\frac{\partial J}{\partial h_1}$$
$$ \frac{\partial J}{\partial W_2} = \frac{\partial J}{\partial net_2} h_1$$
$$ \frac{\partial J}{\partial W_1} = \frac{\partial J}{\partial net_1} x$$

For reference, here is also listed the steps to get $Hv$:

$$ R(net) = R(x)W_1 + xv_1 $$
$$ R(h_1) = act'(net_1)R(net_1) $$
$$ R(net_2) = R(h_1)W_2 + h_1v_2 $$
$$ R(h_2) = act'(net_2)R(net_2) $$
$$ R(\frac{\partial J}{\partial h_2}) = R(h_2) $$
$$ R(\frac{\partial J}{\partial net_2}) = $$ %%%%%%%%%
$$ R(\frac{\partial J}{\partial h_1}) = $$ %%%%%%%%%
$$ R(\frac{\partial J}{\partial net_1}) = $$ %%%%%%%%%
$$ Hv_2 = R(\frac{\partial J}{\partial W_2}) = $$ %%%%%%%%%
$$ Hv_1 = R(\frac{\partial J}{\partial W_1}) = $$ %%%%%%%%%

The functional inverse of the gradient:
%%% backward to error
%%% error to forwards

Finally, what would be $H^{-1}v$:

%%%% backward H^{-1}v
%%%% forward H^{-1}v

\section{Discussion / Strange Speculative Statements}

There are many heterodox views of neural networks, Werbos-Rumelhart-style backpropagation nets. I collect them, and have a few of my own. An important one to me is to think of them as sort of simulations of firms, which also transduce credit through functional layers. Another is to think of them as spins on a 1-lattice with weirdo boundary conditions, each \textit{layer} one \textit{spin} on that small 1-lattice. Note that this is different from strictly going on a Hopfield net view, because of the semantics of what the spin sites are and the boundary conditions.

Second order phenomena are obviously of great interest to both heterodox views. I will present the view I have of how this impinges on them, but don't pretend to have any proof for the view.

In the firm-simulation it seems to be a firm without intrafirm conflict, sort of a communitarian utopian firm, because the taking into account curvature of each parameter with each other parameter seems to coincide with a sort of agreement. This amuses me greatly in that it takes the term "trust region" much more literally than imagined.

In the view as 1-lattice with spins, the inverse Hessian futzes with the exponential gradient correlational structure (the vanishing or exponential gradient) which seems very reminiscent of the onset of the power law of correlations in various critical phenomena.

You might, of course, just care about second-order methods for completely normal optimizational reasons.

\section{Citations}

\begin{thebibliography}{9}
\bibitem{lang1995}
Lang, S. (1995). Differential and Riemannian manifolds, Vol. 160 of. Graduate Texts in Mathematics, 185.
\bibitem{pearlmutter1994}
Pearlmutter, B. A. (1994). Fast exact multiplication by the Hessian. Neural computation, 6(1), 147-160.
\bibitem{deeplearning}
Goodfellow, I., Bengio, Y., Courville, A., \& Bengio, Y. (2016). Deep learning (Vol. 1). Cambridge: MIT press.
\bibitem{martens2010}
Martens, J. (2010, June). Deep learning via Hessian-free optimization. In ICML (Vol. 27, pp. 735-742).
\bibitem{gateaux1913}
G\^{a}teaux, R. (1913). Sur les fonctionnelles continues et les fonctionnelles analytiques. CR Acad. Sci. Paris, 157(325-327), 65.
\end{thebibliography}

\end{document}
